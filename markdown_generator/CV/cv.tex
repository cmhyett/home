%% start of file `template.tex'.
%% Copyright 2006-2015 Xavier Danaux (xdanaux@gmail.com), 2020-2021 moderncv maintainers (github.com/moderncv).
%
% This work may be distributed and/or modified under the
% conditions of the LaTeX Project Public License version 1.3c,
% available at http://www.latex-project.org/lppl/.


\documentclass[11pt,a4paper,sans]{moderncv}        % possible options include font size ('10pt', '11pt' and '12pt'), paper size ('a4paper', 'letterpaper', 'a5paper', 'legalpaper', 'executivepaper' and 'landscape') and font family ('sans' and 'roman')

% moderncv themes
\moderncvstyle{casual}                             % style options are 'casual' (default), 'classic', 'banking', 'oldstyle' and 'fancy'
\moderncvcolor{blue}                               % color options 'black', 'blue' (default), 'burgundy', 'green', 'grey', 'orange', 'purple' and 'red'
%\renewcommand{\familydefault}{\sfdefault}         % to set the default font; use '\sfdefault' for the default sans serif font, '\rmdefault' for the default roman one, or any tex font name
%\nopagenumbers{}                                  % uncomment to suppress automatic page numbering for CVs longer than one page

% character encoding
%\usepackage[utf8]{inputenc}                       % if you are not using xelatex ou lualatex, replace by the encoding you are using
%\usepackage{CJKutf8}                              % if you need to use CJK to typeset your resume in Chinese, Japanese or Korean

% adjust the page margins
\usepackage[scale=0.85]{geometry}
\setlength{\footskip}{136.00005pt}                 % depending on the amount of information in the footer, you need to change this value. comment this line out and set it to the size given in the warning
%\setlength{\hintscolumnwidth}{3cm}                % if you want to change the width of the column with the dates
%\setlength{\makecvheadnamewidth}{10cm}            % for the 'classic' style, if you want to force the width allocated to your name and avoid line breaks. be careful though, the length is normally calculated to avoid any overlap with your personal info; use this at your own typographical risks...

% font loading
% for luatex and xetex, do not use inputenc and fontenc
% see https://tex.stackexchange.com/a/496643
\ifxetexorluatex
  \usepackage{fontspec}
  \usepackage{unicode-math}
  \defaultfontfeatures{Ligatures=TeX}
  \setmainfont{Latin Modern Roman}
  \setsansfont{Latin Modern Sans}
  \setmonofont{Latin Modern Mono}
  \setmathfont{Latin Modern Math} 
\else
  \usepackage[utf8]{inputenc}
  \usepackage[T1]{fontenc}
  \usepackage{lmodern}
\fi

% personal data
\name{Criston}{Hyett}
%\title{Résumé title}                               % optional, remove / comment the line if not wanted
%\born{4 July 1776}                                 % optional, remove / comment the line if not wanted
\address{2525 E Prince Rd, Apt 61}{Tucson, AZ, 85716}{}% optional, remove / comment the line if not wanted; the "postcode city" and "country" arguments can be omitted or provided empty
\phone[mobile]{+1~(409)~201~8902}                   % optional, remove / comment the line if not wanted; the optional "type" of the phone can be "mobile" (default), "fixed" or "fax"
\email{cmhyett@math.arizona.edu}                               % optional, remove / comment the line if not wanted
\homepage{cmhyett.github.io/home/}                         % optional, remove / comment the line if not wanted

% Social icons
\social[github]{cmhyett}                              % optional, remove / comment the line if not wanted
\social[googlescholar]{Criston Hyett}            % optional, remove / comment the line if not wanted

%\extrainfo{additional information}                 % optional, remove / comment the line if not wanted
%\photo[64pt][0.4pt]{picture}                       % optional, remove / comment the line if not wanted; '64pt' is the height the picture must be resized to, 0.4pt is the thickness of the frame around it (put it to 0pt for no frame) and 'picture' is the name of the picture file
%\quote{Some quote}                                 % optional, remove / comment the line if not wanted

% bibliography adjustments (only useful if you make citations in your resume, or print a list of publications using BibTeX)
%   to show numerical labels in the bibliography (default is to show no labels)
%\makeatletter\renewcommand*{\bibliographyitemlabel}{\@biblabel{\arabic{enumiv}}}\makeatother
\renewcommand*{\bibliographyitemlabel}{[\arabic{enumiv}]}
\renewcommand{\listitemsymbol}{}
%   to redefine the bibliography heading string ("Publications")
%\renewcommand{\refname}{Articles}

% bibliography with mutiple entries
%\usepackage{multibib}
%\newcites{book,misc}{{Books},{Others}}
%----------------------------------------------------------------------------------
%            content
%----------------------------------------------------------------------------------
\begin{document}
%\begin{CJK*}{UTF8}{gbsn}                          % to typeset your resume in Chinese using CJK
%-----       resume       ---------------------------------------------------------
\makecvtitle

\section{Fields Of Interest}
\hspace{1in}Physics-Informed Machine Learning, Reduced-Order Modeling, Optimization

\section{Education}
\cventry{2019-2024  (expected)}{Ph.D.}{University of Arizona}{Tucson, AZ}{}{Applied Mathematics}
\cventry{2019-2021}{M.S.}{University of Arizona}{Tucson, AZ}{}{Applied Mathematics}
\cventry{2012-2016}{B.S.}{University of Arizona}{Tucson, AZ}{}{Mathematics \& Physics}

\section{Experience}
\cventry{2021-present}{Graduate Research Assistant}{University of Arizona}{Tucson, AZ}{}{}
\cventry{Summer 2021}{Graduate Student Researcher}{Los Alamos National Labs}{Los Alamos, NM}{}{}
\cventry{2020-2021}{Graduate Research Assistant}{University of Arizona}{Tucson, AZ}{}{}
\cventry{Summer 2020}{Graduate Student Researcher}{Los Alamos National Labs}{Los Alamos, NM}{}{}
\cventry{2019-2020}{Graduate Teaching Assistant}{University of Arizona}{Tucson, AZ}{}{}
\cventry{2016-2019}{Software Engineer II}{Raytheon Missile Systems}{Tucson, AZ}{}{}

\section{Talks}
\cvitem[0cm]{Nov, 2021}{Machine Learning Statistical Evolution of the Coarse-Grained Velocity Gradient Tensor}
\cvitem[0.2cm]{}{\hspace{0.5cm}\textit{APS Division of Fluid Dynamics Meeting}}

\cvitem[0cm]{Mar, 2021}{Machine Learning Stochastic Differential Equations: Applications in Reduced-Order Models of Turbulence}
\cvitem[0.2cm]{}{\hspace{0.5cm}\textit{SIAM Student Brownbag}}

\cvitem[0cm]{Nov, 2020}{Machine Learning Statistical Lagrangian Geometry of Turbulence}
\cvitem[0.2cm]{}{\hspace{0.5cm}\textit{APS Division of Fluid Dynamics Meeting}}

\section{Teaching}
\cvitem{Fall 2019}{Math 112: College Algebra}
\cvitem{Spring 2020}{Math 112: College Algebra}

\section{Fellowships}
\cvitemwithcomment{Jan 2022 - present}{Roots for Resilience Data Science Scholarship}{University of Arizona Data Science Institute, Arizona Institute for Resilience}

\section{Computer Languages}
\cvitemwithcomment{Julia}{Proficient}{Used daily in development of research software}
\cvitemwithcomment{C/C++}{Proficient}{Used extensively in an embedded environment at Raytheon Missile Systems}
\cvitemwithcomment{Bash}{Comfortable}{Basic functionality used daily}
\cvitemwithcomment{Python}{Comfortable}{Used weekly}
\cvitemwithcomment{R}{Beginner}{}
\cvitemwithcomment{Matlab}{Comfortable}{Interpretted monthly}
\cvitemwithcomment{Cuda}{Beginner}{}
\cvitemwithcomment{Ada}{Comfortable}{Interpretted daily while at RMS}

\section{Computer skills}
\cvitem{Open Software}{git, github, \LaTeX}
\cvitem{HPC}{Slurm}
\cvitem{Methodologies}{CI (Jenkins), TDD, Agile}
\cvitem{Operating Systems}{Linux, Windows}

\section{Service and Leadership}
\cvitem[0.2cm]{Aug 2021 - present}{SIAM Brownbag Student Colloquium Organizer}
\cvitem[0.2cm]{Jul 2018 - Jul 2019}{Certified Scrum Master: Scaled Agile Framework}

\section{Interests}
\cvitem[0.2cm]{Reproducible, Interpretable Science}{Using the paradigm of \textit{Literate Programming} and \textit{Test Driven Development} with a minimal toolchain built from Emacs \& git, to co-locate scientific justification with source controlled software, and reproducible results.}

\section{Human Languages}
\cvitemwithcomment{English}{Native Speaker}{}
\cvitemwithcomment{Spanish}{Basic}{}
\cvitemwithcomment{Japanese}{Beginner}{}
\cvitemwithcomment{Amharic}{Beginner}{}

\clearpage
%-----       letter       ---------------------------------------------------------
% recipient data
\recipient{Roots for Resilience in Data Science Scholarship Program}{Arizona Institues for Resilience \& Data Science Institute, University of Arizona}
\date{\today}
\opening{To whom it may concern,}
\closing{I appreciate your consideration,}
\enclosure[Attached]{curriculum vit\ae{}}          % use an optional argument to use a string other than "Enclosure", or redefine \enclname
\makelettertitle

%\textbf{How your interests fit the program + What you would work on under the auspices of the program?}
   I, along with an interdisciplinary group from the University of Arizona and Los Alamos National Labs, work to develop reduced order models of turbulent flows. Turbulence continues to produce some of the largest and most intricate datasets in the world, and our research works to find spatiotemporal patterns that allow a more concise description of this ubiquitous - yet still mysterious - phenomenon. Further, turbulence provides a challenging proving ground for the latest in data science and machine learning.
   My work under this program would be twofold. First, create standardized, parallelizable templates to access the John Hopkins Turbulence Database (JHTDB) and enable the rapid prototyping of data-driven methodologies. JHTDB provides upwards of 430 Terabytes of highly resolved turbulence data along with powerful server-side preprocessing. This combination is an incredible enabler of data science in turbulence, however the interface is feature-rich and naive queries are prohibitively expensive due to the size and complexity of the datasets - I would produce a short document along with efficient templates to shorten the learning curve for researchers wanting to access the database. Second, I plan to dive deep into containers and export the resulting knowledge to my colleagues in the Mathematics department to allow for archive-quality, reproducible, scientific repositories.

   % \textbf{How would the applied math program would benefit?}
   After completion of this work, researchers interested in applying data-driven methodologies to turbulence would have efficient access to a large and vetted database. Further via presentations to my colleagues, I would diffuse experience using containers to create repositories with guaranteed reproducibility.

\makeletterclosing

% Publications from a BibTeX file without multibib
%  for numerical labels: \renewcommand{\bibliographyitemlabel}{\@biblabel{\arabic{enumiv}}}% CONSIDER MERGING WITH PREAMBLE PART
%  to redefine the heading string ("Publications"): \renewcommand{\refname}{Articles}
\nocite{*}
\bibliographystyle{plain}
\bibliography{../publications.bib}                        % 'publications' is the name of a BibTeX file

% Publications from a BibTeX file using the multibib package
%\section{Publications}
%\nocitebook{book1,book2}
%\bibliographystylebook{plain}
%\bibliographybook{publications}                   % 'publications' is the name of a BibTeX file
%\nocitemisc{misc1,misc2,misc3}
%\bibliographystylemisc{plain}
%\bibliographymisc{publications}                   % 'publications' is the name of a BibTeX file

\end{document}


%% end of file `template.tex'.

